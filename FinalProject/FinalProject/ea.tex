\section{Evolutionary Algorithm}
The evolutionary algorithm \ref{alg1} for this program starts off randomly creating a population of two thousand and then evaluating two thousand generations. The population is first initialized and evaluated. The best individual is kept and stored. All individuals equal in fitness are also kept between the generations. After the best fitness is found the all the bad fitness individuals are selected for recombination based on the the best individuals that population. A new generation is created based on the best fitness individuals and the ones recombined based off of their weights. The genetic algorithm takes care of selecting weights to recombine as well as the mutation rate for individual offspring, while the tournament selection handles the evaluation of an individual during the game based on how many times they win and become runner up.

\subsection{Usage}

\begin{quote}

python3 ga.py <population_size> <number_of_iterations> <mutation_rate> <selection_rate>

\end{quote} 

All command line arguments are optional - the default arguments can be seen in the code.  After starting the script, it will run until the number of iterations has been completed.  Note that it will likely enter a state where the games do not end - this is an issue which can be resolved, but was not due to practical reasons.

\begin{algorithm} [tbh]                     % enter the algorithm environment
\caption{Evolve ANNs}          % give the algorithm a caption
\label{alg1}                           % and a label for \ref{} commands later in the document
\begin{algorithmic}                    % enter the algorithmic environment
    \STATE initialize population
    \STATE evaluate population
    \FOR{ i in range( number\_of\_generations ) }
    	\STATE get top fitness
    	\FOR{ p in population }
    		\FOR{k in range( 0 , keep ranks)}
    			\IF{ keep[k] < p.fitness and not (p.fitness in keep) }
    				\STATE keep[k] = p.fitness
    			\ENDIF
    		\ENDFOR
    	\ENDFOR
    	\FOR{ j in range(0, len(population)) }
    		\IF{ not population[j].fitness in keep }
    			\STATE delete population[j]
    		\ENDIF
    	\ENDFOR
    	\STATE Evolve Population using GA
    	\STATE Evaluate Population
    \ENDFOR
\end{algorithmic}
\end{algorithm}

\subsection{Genetic Algorithm}
The genetic algorithm takes the best population based on a threshold in ea and creates offspring with combinations of the weights of the best individuals. It also takes in a a mutation rate parameter that will randomly select that number of weights and randomly assign them a new value. The algorithm is:
\begin{algorithm} [tbh]                     % enter the algorithm environment
\caption{GA}          % give the algorithm a caption
\label{GA}                           % and a label for \ref{} commands later in the document
\begin{algorithmic}                    % enter the algorithmic environment
    \STATE initialize children
    \FOR{ i in range( 0, n ) }%number\_of\_generations ) }
    	\STATE a = randrange( 0, len(population)) 
    	\STATE b = randrange( 0, len(population))
    	\WHILE {a == b}
    		\STATE b = randrange( 0, len(population )) 
    	\ENDWHILE
    	\STATE child = create\_ann()
    	\STATE childe.combine\_weights(population[a].weights, population[b].weights)
    	\IF{ random.random() < mutation\_rate }
    		\STATE child.mutate\_weight(1)
    	\ENDIF
    	\STATE children.append(child)
    \ENDFOR
    \STATE return children
\end{algorithmic}
\end{algorithm}
\subsection{Tournament Selection}
The tournament selection is where each artificial network is played against other networks. These games are done with 3 players that are randomly chosen each time a new game starts. ANN's won't play themselves in these games. The amount of times a ANN comes in first is recorded, as well as the number of times it is a runner-up. After all the games are played the fitness of each ANN is calculated by the number of times it was first + runner-up counts all divided by the max of the runner-up counts. The algorithm is as follows:
\begin{algorithm} [tbh]                     % enter the algorithm environment
\caption{Tournement Selection Algorithm}          % give the algorithm a caption
\label{Tourny}                           % and a label for \ref{} commands later in the document
\begin{algorithmic}                    % enter the algorithmic environment
    \STATE initialize win\_counts[]
    \STATE initialize runner\_up
    \FOR{ i in range( 0, max\_games ) }%number\_of\_generations ) }
    	\STATE Select 3 random individuals and Make sure none are the same
    	\STATE Play ANNs against each other in Game
    	\STATE increment wincounts and runner-up counts for each ANN    	
    \ENDFOR
    \FOR{ i in range( 0, len(anns) ) }
    	\STATE anns[k].fitness = wincounts[k] + runnerupcounts[k]/max(runnerupcounts[k])
    \ENDFOR
\end{algorithmic}
\end{algorithm}